\documentclass[11pt]{article}
\usepackage[top=1in, bottom=1in, left=1.25in, right=1.25in]{geometry}
\usepackage[BoldFont,SlantFont,CJKsetspaces,CJKchecksingle]{xeCJK}
\usepackage{framed}
\usepackage{amsmath}
\usepackage{listings}
\usepackage{hyperref}
\setCJKmainfont[BoldFont=SimHei]{SimSun}
\setCJKmonofont{SimSun}
\usepackage{graphicx}
\usepackage{color}
\definecolor{mygreen}{rgb}{0,0.6,0}
\definecolor{mygray}{rgb}{0.5,0.5,0.5}
\definecolor{mymauve}{rgb}{0.58,0,0.82}
\lstset{ %
  backgroundcolor=\color{white},
  basicstyle=\ttfamily,
  breakatwhitespace=false,
  breaklines=true,
  captionpos=t,
  commentstyle=\color{mygreen},
  columns=flexible,
  deletekeywords={...},
  escapeinside={\%*}{*)},
  extendedchars=true,
  frame=single,
  keepspaces=true,
  keywordstyle=\color{blue},
  language=C,
  morekeywords={*,...},
  numbers=none,
  rulecolor=\color{black},
  showspaces=false,
  showstringspaces=false,
  showtabs=false,
  stepnumber=1,
  stringstyle=\color{mymauve},
  tabsize=2,
  title=\lstname
}
\newenvironment{packed_enum}{
\begin{enumerate}
  \setlength{\itemsep}{1pt}
  \setlength{\parskip}{0pt}
  \setlength{\parsep}{0pt}
}{\end{enumerate}}
\parindent 2em

\begin{document}
\title{\textbf{\huge{实验报告 Lab5}}\\
10300240039 章超}
\maketitle
\section{File system preliminaries}
第一部分,我们首先要使JOS拓展称为多处理器的系统,然后实现新的系统调用,从而使用户可以创建进程(JOS的Enviroment),进程调度将使用round-robin轮转式调度,在第三部分中再实现抢占式调度。
\subsection{Disk Access}

\begin{framed}
\noindent\textbf{Exercise 1} \lstinline|i386_init| identifies the file system environment by passing the type \lstinline|ENV_TYPE_FS| to your environment creation function, \lstinline|env_create|. Modify \lstinline|env_create| in \lstinline|env.c|, so that it gives the file system environment I/O privilege, but never gives that privilege to any other environment.

Make sure you can start the file environment without causing a General Protection fault. You should pass the "fs i/o" test in \lstinline|make grade|.
\end{framed}

\begin{framed}
\noindent\textbf{Question 1} Do you have to do anything else to ensure that this I/O privilege setting is saved and restored properly when you subsequently switch from one environment to another? Why?
\end{framed}

\subsection{The Block Cache}
\begin{framed}
\noindent\textbf{Exercise 2} Implement the \lstinline|bc_pgfault| functions in \lstinline|fs/bc.c|. \lstinline|bc_pgfault| is a page fault handler, just like the one your wrote in the previous lab for copy-on-write fork, except that its job is to load pages in from the disk in response to a page fault. When writing this, keep in mind that (1) \lstinline|addr| may not be aligned to a block boundary and (2) \lstinline|ide_read| operates in sectors, not blocks.

Use \lstinline|make grade| to test your code. Your code should pass "check_super".
\end{framed}

\subsection{The file system interface}

\section{Spawning Processes}
\begin{framed}
\noindent\textbf{Exercise 3} \lstinline|spawn| relies on the new syscall \lstinline|sys_env_set_trapframe| to initialize the state of the newly created environment. Implement \lstinline|sys_env_set_trapframe|. Test your code by running the \lstinline|user/spawnhello| program from \lstinline|kern/init.c|, which will attempt to spawn \lstinline|/hello| from the file system.

Use \lstinline|make grade| to test your code.
\end{framed}

\subsection{Sharing library state across fork and spawn}
\begin{framed}
\noindent\textbf{Exercise 4} Change duppage in \lstinline|lib/fork.c| to follow the new convention. If the page table entry has the \lstinline|PTE_SHARE| bit set, just copy the mapping directly. (You should use \lstinline|PTE_SYSCALL|, not \lstinline|0xfff|, to mask out the relevant bits from the page table entry. \lstinline|0xfff| picks up the accessed and dirty bits as well.)

Likewise, implement \lstinline|copy_shared_pages| in \lstinline|lib/spawn.c|. It should loop through all page table entries in the current process (just like fork did), copying any page mappings that have the \lstinline|PTE_SHARE| bit set into the child process.
\end{framed}

\section{The keyboard interface}
\begin{framed}
\noindent\textbf{Exercise 5} In your \lstinline|kern/trap.c|, call \lstinline|kbd_intr| to handle trap \lstinline|IRQ_OFFSET+IRQ_KBD| and \lstinline|serial_intr| to handle trap \lstinline|IRQ_OFFSET+IRQ_SERIAL|.
\end{framed}

\section{The Shell}
\begin{framed}
\noindent\textbf{Question 2} How long approximately did it take you to do this lab?
\end{framed}

\begin{framed}
\noindent\textbf{Question 3} We simplified the file system this year with the goal of making more time for the final project. Do you feel like you gained a basic understanding of the file I/O in JOS? Feel free to suggest things we could improve.
\end{framed}

\section{Challenges}


\section{Tips}
\begin{packed_enum}
\item Kernel调试信息可以用warn输出,User Mode则要用cprintf
\end{packed_enum}

\section{References}
北京大学操作系统实习(实验班)报告,黄睿哲。
操作系统JOS实习第一次报告,张弛。
\end{document}
